SOMEBODY ASKED A ZEN MASTER, 'WHAT IS THE GREATEST MIRACLE
IN THE WORLD?' THE MASTER REPLIED, 'I AM SITTING HERE ALONE
WITH ME.'

WHAT IS THE MEANING OF THIS PARABLE?

IT IS NOT A PARABLE, IT IS SIMPLY A FACT. Look directly into it. There is no need to search for any meaning. It is like a rose flower - a simple statement. If you start looking for meaning you will miss the meaning of it. The meaning is there, obvious; there is no need to search for it. The moment you start searching for meaning about such simple facts, you weave philosophies, you create metaphysics. And then you go on and on, and you go far away from the fact.

It is a simple statement. The zen master said, 'I am sitting here alone with myself.' This is the greatest miracle. To be alone is the greatest achievement. One feels always a need for the other. There is a tremendous need for the other because something is lacking within ourselves. We have holes in our being; we stuff those holes with the presence of the other. The other somehow makes us complete, otherwise we are incomplete.

Without the other we don't know who we are, we lose our identity. The other becomes a mirror and we can see our faces in it. Without the other we are suddenly thrown to ourselves. Great uncomfort, inconvenience arises, because we don't know who we are. When we are alone we are in very strange company, very embarrassing company. We don't know with whom we are.

With the other, things are clear, defined. We know the name, we know the form, we know the man or the woman - Hindu, Christian, Indian, American - there are some ways to define the other. How to define yourself?

Deep down there is an abyss... undefinable. There is an abyss... emptiness. You start merging into that. It creates fear. You become frightened. you want to rush towards the other. The other helps you to hang out, the other helps you to remain out. When there is nobody you are simply left with your emptiness.

Nobody wants to be alone. The greatest fear in the world is to be left alone.

People do a thousand and one things just not to be left alone. You imitate your neighbours so you are just like them and you are not left alone. You lose your individuality, you lose your uniqueness, you just become imitators, because if you are not imitators you will be left alone.

You become part of the crowd, you become part of a church, you become part of an organization. Somehow you want to merge with a crowd where you can feel at ease, that you are not alone, there are so many people like you - so many Mohammedans like you, so many Hindus like you, so many Christians, millions of them... you are not alone.

To be alone is really the greatest miracle. That means now you don't belong to any church, you don't belong to any organization, you don't belong to any theology, you don't belong to any ideology - socialist, communist, fascist, hindu, christian, jain, buddhist - you don't belong, you simply are. And you have learnt how to love your indefinable, ineffable reality. You have come to know how to be with yourself.

Your needs for the other have disappeared. You don't have any loopholes, you don't have any holes, you are not missing anything, you don't have any flaws - you are simply happy by being yourself. You don't need anything, your bliss is unconditional. Yes, it is the greatest miracle in the world.

But remember, the master says, 'I am here alone with myself.' When you are alone you are not alone, you are simply lonely - and there is a tremendous difference between loneliness and aloneness. When you are lonely you are thinking of the other, you are missing the other. Loneliness is a negative state.

You are feeling that it would have been better if the other was there - your friend, your wife, your mother, your beloved, your husband. It would have been good if the other was there, but the other is not.

Loneliness is absence of the other. Aloneness is the presence of oneself.

Aloneness is very positive. It is a presence, overflowing presence. You are so full of presence that you can fill the whole universe with your presence and there is no need for anybody.

If the whole world disappears this zen master will not miss anything. If suddenly by some magic the whole world disappears and this zen master is left alone, he will be as happy as ever, he will not miss anything. He will love that tremendous emptiness, this pure infinity. He will not miss anything because he has arrived home. He knows that he himself is enough unto himself.

This does not mean that a man who has become enlightened and has come home does not live with others. In fact only he is capable of being with others. Because he is capable of being with himself he becomes capable of being with others. If you are not capable of being with yourself, how can you be capable of being with others? You are at the closest quarters. Even with yourself you are not capable of being in deep love, in delight - how can you be with others? Others are far away.

A man who loves his aloneness is capable of love, and a man who feels loneliness is incapable of love. A man who is happy with himself is full of love, flowing. He does not need anybody's love, hence he can give. When you are in need how can you give? You are a beggar. And when you can give, much love comes towards you. It is a response, a natural response. The first lesson of love is to learn how to be alone.

It is a very significant statement. It has nothing like a parable in it. It is immediate, direct. It is like a rose flower encountering you. You never ask about a rose flower, 'What is the parable of this rose flower?' You don't ask, 'What is the meaning of this rose flower?'

A master is like a rose flower. If you can see, see. If you cannot see, forget. You will never be able to know its meaning because the meaning is just in front of you. Don't make a parable out of it. Parables mean you have started interpreting, and whatsoever you interpret is going to be your interpretation.

I have heard:

Mulla Nasrudin was caught fishing at a place where there was a big sign: No Fishing Here. The warden who caught him asked, 'Nasrudin, can't you see the sign? Can't you read? - No fishing here.' He pointed to the sign.

Mulla Nasrudin said, 'Yes, I can read, but I don't agree. There is good fishing here. Who says "no fishing here". There is good fishing here. Just look at this lot I have landed today. Whoever put that sign up must be crazy.'

Now this is your interpretation. It is a simple sign - No Fishing Here. The meaning is not to be found, it is simply there.

When a zen master says something, or when any master says something, his meaning is absolutely clear, obvious. It is just in front of you. Don't try to avoid it. If you start looking for meaning you will look left and right and you will miss that which is in front of you. It is a simple statement: 'I am sitting here alone with me.'

Try it, to have the feel. Just sit alone sometimes. That's what meditation is all about - just sitting alone, doing nothing, Just try. If you start feeling lonely then there is something missing in your being, then you have not been able yet to understand who you are.

Then go deeper into this loneliness until you come to a layer when suddenly loneliness transforms itself into aloneness. It transforms - it is a negative aspect of the same phenomenon. Loneliness is the negative aspect of aloneness. If you go deeper into it one moment is bound to come when suddenly you will start feeling the positive aspect of it. Because both aspects are always together.

So be lonely, suffer loneliness. It is difficult, meditation is difficult. People come to me and they ask, 'Yes, we are ready to sit, but give us a mantra so that we can chant a mantra.' What are they asking? They are saying that they don't want to be alone, they don't want to face their loneliness. They will chant a mantra - the mantra will become their companion. They will say, 'Ram, Ram, Ram' - now they are not alone. Now this sound of 'Ram' continuously repeated will become their companion.

They are missing the whole point. Transcendental meditation, TM, is not meditation at all, because meditation simply means to be alone, not doing anything - not even chanting a mantra. Because this is a trick of the mind. That's what the mind has always been doing. When you sit alone, have you watched how many fantasies reveal themselves to you?... endless fantasies, daydreams.

Whenever you are alone, you start daydreaming. Whenever you don't have anything to do and you feel bored, immediately you escape into daydreams.

That's why if a person goes to the desert, to the Arabian Desert, to the Sahara, and sits there, he will start imagining, visions will start coming to him, because a desert is a very monotonous thing. Nothing to pay attention to - just the same monotonous expanse of sands and sands; nothing to distract, nothing new - monotonous, boring. A person becomes dreamy, one starts substituting. If there is nothing new outside, one creates one's own imaginative world and starts looking into it.

That's what happens to people who go to the Himalayas and sit in caves to meditate. They start imagining. Then they can imagine anything - gods and goddesses and apsaras and angels and Krishna playing on his flute, and Rama standing with his bow, and Jesus - and whatsoever your imagination, whatsoever your conditioning. If you have been conditioned as a Christian, sooner or later in a himalayan cave you will encounter Jesus, and this will be pure imagination. Nothing to distract the mind outside, the mind starts creating its own dreams inside. And when you continuously dream, those dreams look very very real.

Many experiments have been done in the West on sense deprivation. If a person is deprived of all impressions - his eyes are closed, he is put in a box, his ears are closed, his whole body is encased in foam rubber so the touch is monotonous, the darkness in the eyes is monotonous, the soundlessness is monotonous, everything monotonous - within two, three hours he starts dreaming - such fantastic dreams, and so real... realer than real. And if a person is deprived for twenty-one days he will never come back sane. He will become insane, because his imagination will take complete possession of him.

But why does the mind start daydreaming? The scientific explanation is that the mind cannot live alone with itself. So either it needs somebody in reality, or, if in reality somebody is not there, then it creates fantasy. Fantasy is a substitute. The mind cannot live alone.

That's why you dream in the night - because in sleep you are alone; the world disappears. Your husband is no more there, your children are no more there, your wife is no more there, you are simply alone - and you have become incapable of aloneness. Your mind simply substitutes another world of dreams; dreams, cycles of dreams the whole night. Why are dreams needed? Because you cannot be alone.

This whole illusion that exists around you is because you have not learned one basic thing - of being alone. The zen master is right. He says, 'This is the greatest miracle. I sit here alone with myself.' To be with oneself and to be happily with oneself, blissfully with oneself, and not to move into fantasies... then suddenly one is at home, one is entering into one's own abyss.

It appears like emptiness when you enter, but once you have entered it is the very fullness of being, the fulfillment, the blossoming, the climax, the crescendo.

It is not emptiness. It only appears to be emptiness because you have lived with others and suddenly you miss the others; that's why you interpret it as empty.

Others are not there, only you are there - but you cannot see yourself right now, you simply miss the others.

You have become too habitual; the idea of the other has become very ingrained, it has become a mechanical habit, so when you miss it you feel you are empty, lonely, falling in an abyss. But if you allow and fall into the abyss, soon you will realize the abyss has disappeared, and with the abyss all the illusory attachments have disappeared. Then happens the greatest miracle - that you are simply happy for no reason at all.

Remember, when your happiness depends on others, your unhappiness also will depend on others. If you are happy because a woman loves you, you will become unhappy if she does not love you. If you are happy for any reason whatsoever, then any day the reason is not there, you will become unhappy. Your happiness will always be on the rocks, you will always remain in stormy weather. You will never be certain whether you are happy or unhappy, because each moment you will see the ground underneath can disappear - any moment it can disappear.

You can never be certain. The woman was smiling just now, and then she has become angry. The husband was talking so beautifully and suddenly he has lost his temper.

Depending on others is depending - it is a bondage, it is a dependence, and one can never feel really blissful.

Blissfulness is possible only in total, unconditional freedom. That's why in the East we call it moksha. Moksha means absolute freedom. To be with oneself is moksha because now you don't depend. Your happiness simply is your own, you don't borrow it from anybody. Nobody can take it away, not even death.

Remember, death only separates you from others, it never separates you from yourself. Death seems so frightening because it will snatch you away from others - the wife will not be any more with the husband, the mother will not be any more with the children. Death only separates you from others. It cannot separate you from yourself; there is no way to separate you from yourself.

Once you have learned how to be with yourself then death is meaningless, then death does not exist. You become deathless. Then death cannot take anything away from you. That which death can take away from you, you have surrendered on your own accord.

That's what meditation is - to surrender the non-essential, that which death can take away from you. That which death is going to do, a meditator does on his own accord, voluntarily. Knowing it well - that this will be taken away - he surrenders it.

It is immensely beautiful to be alone. There is nothing to be compared with it. Its beauty is the ultimate beauty, its grandeur is the ultimate grandeur, its power is the ultimate power.

Come back home. And the way is: you will have to suffer loneliness first. Suffer it, go through it. You have to pay for the bliss that is going to be yours - you have to pay for it. This suffering of loneliness is just paying for it. You will be tremendously benefited.