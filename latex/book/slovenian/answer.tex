        

   Zmeraj cutis potrebo po druzbi. Velikanska je tvoja potreba po druzbi, nekaj manjka tvoji exsistenci. V sebi imas existencne luknje; te luknje polnes z prisotnostjo drugih. Drugi nekako izspopolnijo tvoj obstoj, drugace pa samemu sebi nisi nikoli dovolj. 

Brez drugih neves kdo si, izgubis svojo identiteto. Drugi postanejo tvoje ogledalo v katerem lahko vidis svoj obraz. Brez drugih si kar naenkrat prepuscen samemu sebi. Zacutis nelagodje, paniko, saj v resnici ne ves kdo si. Ko si v samoti si v zelo cudni druzbi, neprijetni, mucni druzbi. Ne ves skom si. 

V druzbi, so tvoje lastnosti jasno definirane. Poznas ime, poznas obliko, poznas moskega ali zensko - Kristjan, Musliman, Slovenec. Ali v resnici ves kdo si ti ko ni nobenega zraven? 

Globoko v sebi cutis nicnost... nedolocljivost. Tam ni nicesar... praznina. Zacnes se zlivati v to nedolocljivo nicnost. Zacutis strah. Postanes na smrt prestrasen. Hitis v druzbo drugega. Prisotnost drugega ti pomaga to pozabiti, drugi ti pomaga da tega vec ne cutis. Ko ni nobenega vec ob tebi si spet prepuscen svoji praznosti. 

Nihce ne zeli ostati sam. Najbolj se bojis osamljenosti. 

Delas tisoc in eno stvar samo da ne bi ostal v osamljenosti. Imitiras bliznje ljudi da si podoben njim in nisi vec osamljen. Izgubis svojo individualnost, izgubis svojo unikatnost, postanes imitator, ker ce nisi imitator ostanes v osamljenosti. 

Postanes del druzbe, postanes clan cerkve, postanes del organizacije. Zelis se izgubiti v druzbi kjer se lahko sprostis ter pozabis na svojo osamljenost, toliko podobnih ljudi tebi - enako mislecih ljudi, na miljone... nisi vec osamljen. 

Ostati sam z sabo je resnicno najvecji cloveski cudez. To pomeni da ne pripadas nobeni cerkvi, ne pripadas nobeni organizaciji, ne pripadas nobeni teologiji, ne pripadas nobeni ideologiji, ne pripadas nobenemu cloveku - ne pripadas nicemur, preposto obstajas brez razloga, samo si. Takrat si se naucil, kako ljubiti svojo nedefinirano, nedolocljivo realnost. Spoznal si kako biti sam sebi najboljsi prijatelj. 

Potrebe po druzbi drugega so izginile. Nimas vec nobenih miselnih vozlov, nimas vec nobenih eksistencnih lukenj, ne pogresas vec nicesar, nimas vec manjkajocih delov - srecen si da lahko preprosto obstajas. Ne potrebujes nicesar vec, tvoja sreca je brez pogojev. Da, to je resnicno najvecji cloveski cudez. 

Toda ne pozabi, mojster pravi, \textquote{Jaz sedim sam s seboj.} Ko si sam s seboj nisi v samoti, si samo osamljen - in tu je ogromna razlika med osamljenostjo in samoto. Ko si osamljen mislis na drugega, pogresas prisotnost drugega. Osamljenost je negativno stanje. 

V osamljenosti cutis da bi bilo bolje ce bi zraven tebe bil se nekdo drug - tvoj prijatelj, tvoja druzina, ljubljena oseba. Ko si osamljen bi bilo lepo ce bi bil zraven se nekdo drug, vendar ni nobenega. 

Osamljenost pomeni pomanjkanje drugega. Samota pomeni prisotnost samega sebe. 

Samota je zelo pozitivna. Samota je prisotnost, zapolnjena, kipeca, prisotnost. Tako si poln prisotnosti da lahko z njo napolnis celotno vesolje, in tam ni potrebe po nobeni druzbi. 

Ce vse izgine, zen mojster ne bo pogresal nicesar. Ce se kar naenkrat zaradi neke carovnije vse izbrise iz vesolja da ostane samo nicelnost in zen mojster, se zaradi tega dogodka njegova sreca ne bo nic spremenila, nicesar ne bo pogresal. Ljubil bo to neskoncno nicelnost, to cisto neskoncnost. Nicesar ne bo pogresal ker je prisel domov. Ve da je on, sam sebi popolnoma dovolj, to je vse kar v resnici rabi. 

To ne pomeni da moz ki je postal razsvetljen in ki je prisel domov ne zna ziveti z drugimi. V resnici je samo on sposoben ziveti v druzbi drugih. Ker je sposoben biti sam s seboj je sposoben biti z drugimi. Ce nisi sposoben biti sam z seboj, kako si lahko sposoben biti z drugimi? Samo ti ves kaj v resnici TI potrebujes. Ce nisi sposoben biti, v globoki ljubezni sam z saboj, v radosti - kako si lahko z drugimi? Drugi lahko samo ugibajo kaj so tvoje resnice potrebe. 

Clovek ki ljubi svojo samoto je sposoben ljubezni, clovek ki cuti osamljenost ni sposoben ljubezni. Clovek ki je srecen sam z seboj je poln ljubezni, radosti. Ne potrebuje ljubezni od nikogar, a se vedno jo lahko daje drugemu. Ko si v pomanjkanju kako lahko dajes kar koli drugim? Ko si v pomanjkanju si berac. Ko lahko dajes drugim, veliko ljubezni dobis nazaj. To je naravni odziv drugega. Prva lekcija ljubezni je da se naucis ziveti v samoti. 

Probaj jo kdaj obcutiti. Probaj sedeti sam z seboj obcasno. To je bistvo meditacije - samo sedeti sam z seboj, delati nicesar, probaj jo kdaj izskusiti. Ce bom zacutil osamljenost potem nekaj manjka tvojemu obstoju, potem se nisi sposoben razumeti kdo si v resnici. 

Pojdi globje v osamljenost dokler ne prides do tocke kjer se osamljenost transformira v samoto. Transformira se - ker je nasprotni, obratni vidik enakega pojava. Osamljenost je nasprotni vidik samote. Ce bos sel globje v osamljenost bo moment zagotovo prisel ko bos zacel cutiti pozitivene vidike osamljenosti. Ker so se tako nasprotne reci zmeraj tesno povezane skupaj. Dan ne obstaja brez noci kakor tudi osamljenost ne obstaja brez samote. 

Torej bodi osamljen, trpi osamljenost. Seveda ti bo tezsko, meditacija je tezska. Ljudje pridejo do mene in vprasajo, \textquote{Da, smo pripravljeni sedeti, toda daj nam mantro da lahko mantramo.} Kaj me vprasajo? Pravijo da nocejo biti sami s seboj, nocejo se soociti z svojo osamljenostjo. Oni bodo mantrali mantro - mantra bo postala njihova druzba. Mantrali bodo, \textquote{Ohm, Ohm, Ohm} - in ne bodo vec osamljeni. Ta zvok \textquote{Ohm} neprestalno ponovljen bo postal njihov druzabnik. 

Bistvo meditacije so zaradi tega izgubili. Meditacija z mantranjem, sploh ni meditacija, ker meditacija enostavno pomeni biti sam brez aktivnosti - niti mantrati mantro. To je trik uma. To um zmeraj dela. Ko sedis v samoti, ali si kdaj videl koliko fantazij zacnes videvati?... neskoncne fantazije, sanje. 

Kadarkoli si sam, zacnes sanjariti. Ko nimas nicesar za delati in zacutis dolgocasje, takoj se zacenjas zatekati v sanjarjenje. 

That's why if a person goes to the desert, to the Arabian Desert, to the Sahara, and sits there, he will start imagining, visions will start coming to him, because a desert is a very monotonous thing. Nothing to pay attention to - just the same monotonous expanse of sands and sands; nothing to distract, nothing new - monotonous, boring. A person becomes dreamy, one starts substituting. If there is nothing new outside, one creates one's own imaginative world and starts looking into it. 

That's what happens to people who go to the Himalayas and sit in caves to meditate. They start imagining. Then they can imagine anything - gods and goddesses and apsaras and angels and Krishna playing on his flute, and Rama standing with his bow, and Jesus - and whatsoever your imagination, whatsoever your conditioning. If you have been conditioned as a Christian, sooner or later in a himalayan cave you will encounter Jesus, and this will be pure imagination. Nothing to distract the mind outside, the mind starts creating its own dreams inside. And when you continuously dream, those dreams look very very real. 

Many experiments have been done in the West on sense deprivation. If a person is deprived of all impressions - his eyes are closed, he is put in a box, his ears are closed, his whole body is encased in foam rubber so the touch is monotonous, the darkness in the eyes is monotonous, the soundlessness is monotonous, everything monotonous - within two, three hours he starts dreaming - such fantastic dreams, and so real... realer than real. And if a person is deprived for twenty-one days he will never come back sane. He will become insane, because his imagination will take complete possession of him. 

But why does the mind start daydreaming? The scientific explanation is that the mind cannot live alone with itself. So either it needs somebody in reality, or, if in reality somebody is not there, then it creates fantasy. Fantasy is a substitute. The mind cannot live alone. 

That's why you dream in the night - because in sleep you are alone; the world disappears. Your husband is no more there, your children are no more there, your wife is no more there, you are simply alone - and you have become incapable of aloneness. Your mind simply substitutes another world of dreams; dreams, cycles of dreams the whole night. Why are dreams needed? Because you cannot be alone. 

This whole illusion that exists around you is because you have not learned one basic thing - of being alone. The zen master is right. He says, \textquote{This is the greatest miracle. I sit here alone with myself.} To be with oneself and to be happily with oneself, blissfully with oneself, and not to move into fantasies... then suddenly one is at home, one is entering into one's own abyss. 

It appears like emptiness when you enter, but once you have entered it is the very fullness of being, the fulfillment, the blossoming, the climax, the crescendo. 

It is not emptiness. It only appears to be emptiness because you have lived with others and suddenly you miss the others; that's why you interpret it as empty. 

Others are not there, only you are there - but you cannot see yourself right now, you simply miss the others. 

You have become too habitual; the idea of the other has become very ingrained, it has become a mechanical habit, so when you miss it you feel you are empty, lonely, falling in an abyss. But if you allow and fall into the abyss, soon you will realize the abyss has disappeared, and with the abyss all the illusory attachments have disappeared. Then happens the greatest miracle - that you are simply happy for no reason at all. 

Remember, when your happiness depends on others, your unhappiness also will depend on others. If you are happy because a woman loves you, you will become unhappy if she does not love you. If you are happy for any reason whatsoever, then any day the reason is not there, you will become unhappy. Your happiness will always be on the rocks, you will always remain in stormy weather. You will never be certain whether you are happy or unhappy, because each moment you will see the ground underneath can disappear - any moment it can disappear. 

You can never be certain. The woman was smiling just now, and then she has become angry. The husband was talking so beautifully and suddenly he has lost his temper. 

Depending on others is depending - it is a bondage, it is a dependence, and one can never feel really blissful. 

Blissfulness is possible only in total, unconditional freedom. That's why in the East we call it moksha. Moksha means absolute freedom. To be with oneself is moksha because now you don't depend. Your happiness simply is your own, you don't borrow it from anybody. Nobody can take it away, not even death. 

Remember, death only separates you from others, it never separates you from yourself. Death seems so frightening because it will snatch you away from others - the wife will not be any more with the husband, the mother will not be any more with the children. Death only separates you from others. It cannot separate you from yourself; there is no way to separate you from yourself. 

Once you have learned how to be with yourself then death is meaningless, then death does not exist. You become deathless. Then death cannot take anything away from you. That which death can take away from you, you have surrendered on your own accord. 

That's what meditation is - to surrender the non-essential, that which death can take away from you. That which death is going to do, a meditator does on his own accord, voluntarily. Knowing it well - that this will be taken away - he surrenders it. 

It is immensely beautiful to be alone. There is nothing to be compared with it. Its beauty is the ultimate beauty, its grandeur is the ultimate grandeur, its power is the ultimate power. 

Come back home. And the way is: you will have to suffer loneliness first. Suffer it, go through it. You have to pay for the bliss that is going to be yours - you have to pay for it. This suffering of loneliness is just paying for it. You will be tremendously benefited. 