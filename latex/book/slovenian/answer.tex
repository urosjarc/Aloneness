        

   Zmeraj čutiš potrebo po družbi. Potreba je ogromna, nekaj ti manjka. V sebi imaš luknje, polniš jih z drugimi ljudmi. Oni te nekako izpopolnijo, drugače pa si nepopoln. 

Brez družbe ne veš, kdo si, izgubiš svojo identiteto. Oni so tvoje ogledalo, v katerem se lahko gledaš. Brez družbe si kar naenkrat prepuščen samemu sebi, takrat začutiš nelagodje, ker ne veš, kdo si. Ko si sam, si v zelo neprijetni, mučni družbi. Ne veš, s kom si, ker ne veš, kdo si. 

Z drugo stvarjo si si zgradil svoj jaz, svojo osebnost. Veš, komu pripadaš, služiš, kaj ljubiš, sovražiš, veruješ. Z drugo stvarjo si si definiral svoj jaz, zato točno veš, kdo si! Kdo si, ko zraven tebe ni ničesar? 

Globoko v sebi čutiš praznino ... nedoločljivost. Tam je luknja ... ničnost. Njena gravitacija te vleče vanjo. Začutiš strah. Tvoj jaz se izgublja v ničnosti, postaneš paničen. Hitiš v družbo. Oni te rešijo pekla. Spet čutiš svoj jaz. Zdaj ga spet lahko vidiš! Ko si sam, si spet prepuščen svoji praznini, ničnosti. 

Nihče si ne želi biti sam. Tvoj največji strah je, da bi nekoč ostal sam. 

Narediš tisoč in eno stvar, samo da ne bi ostal sam. Posnemaš druge, da si jim všeč, da se družijo s tabo, da nisi več osamljen. Izgubiš svojo individualnost, unikatnost. Postaneš imitator, ker bi drugače ostal sam. 

Postaneš del družine, organizacije, religije, ideologije. Želiš postati del nečesa, da se lahko sprostiš, da lahko odmisliš svojo osamljenost. Toliko \textquote{podobnih}, enako mislečih ljudi obstaja. Zraven njih nisi več osamljen. 

Biti sam je res največji čudež. To pomeni, da ne pripadaš nobenemu človeku, nobeni družini, nobeni organizaciji, nobeni religiji, nobeni ideologiji, nobenemu cilju. Ne pripadaš nikomur, ne pripadaš ničemur. Ne pripadaš ... samo obstajaš, samo si. Takrat si se naučil, kako ljubiti svojo nedefinirano, nedoločljivo osebnost. Tvoj jaz ni več definiran z zunanjimi stvarmi. Spoznal si, kako biti sam s seboj. 

Potrebe po družbi so izginile. Nimaš več miselnih vojn, nimaš več eksistenčnih kriz identitete, ničesar več ne pogrešaš. Nimaš več pomanjkljivosti ... srečen si sam s seboj. Ne potrebuješ ničesar več, zdaj je tvoja sreča brez\-po\-goj\-na. Da, to je največji čudež. 

Toda mojster pravi: \textquote{Jaz sedim tu sam s seboj.} Ko si sam, nisi sam, si samo osamljen. Razlika med osamljenostjo in samoto je ogromna. Ko si osamljen, pogrešaš drugo stvar, takrat pogrešaš družbo. Osamljenost je negativno, ne-čuječnostno umsko stanje. 

Čutiš, razmišljaš, da bi bilo bolje, če bi bilo zraven tebe še nekaj - prijatelj, družina, ljubljena oseba, glasba, narava, vznemirjenje, drama, akcija ... karkoli ... Bilo bi lepo, če bi bilo zraven tebe, a tega ni. 

Ko si osamljen, si jokajoč otrok, ki si želi nekaj, česar trenutno nima v svojih rokah. Osamljenost je pogrešanje, samota pa je prisotnost samega sebe. 

Samota je pozitivno, zelo čuječnostno umsko stanje. Samota je opazovanje samega sebe, izkustvo sebe v tem trenutku. Druženje s samim seboj.  S samim seboj lahko napolniš svoje lastne luknje. S samim seboj lahko napolniš prostor, kjer bi moral biti drugi. Cela vesolja lahko nasičiš s samim seboj in takrat ne pogrešaš ničesar več. 

Če bi ves svet izginil, ne bi v tej praznini pogrešal ničesar.  Ljubil bi to velikansko praznino, čisto neskončnost, v kateri bi bil ti. Ničesar ne bi pogrešal, ker bi takrat prišel domov. Vedel bi, da si, sam sebi dovolj, da je to vse, kar v resnici potrebuješ. 

To pa še ne pomeni, da mož, ki zna živeti sam, ne zna živeti z drugimi. V resnici je samo on sposoben živeti z drugimi. Ker zna živeti sam s sabo, zna živeti z drugimi. Če nisi sposoben biti sam s seboj, kako si lahko sposoben živeti z drugimi? Samo TI poznaš sebe najbolje. Če s samim sabo ne znaš živeti v globoki ljubezni, ekstazi, kako lahko pričakuješ od drugih, ki te ne poznajo, kot sam poznaš sebe, da ti bodo dajali to, kar si TI še sam sebi ne znaš dati? Drugi ne ve! Drugi ne ve. Drugi ne ve ... 

Človek, ki ljubi svojo samoto, je sposoben ljubezni, človek, ki čuti osam\-ljenost, pa je ni. Človek, ki je srečen sam s seboj, je poln ljubezni. Ne potrebuje ljubezni od kogarkoli, a še vedno jo daje. Ko si v pomanjkanju nečesa, kako lahko daješ to drugemu? Ko si v pomanjkanju, si berač. Ko imaš ljubezen v neomejenih količinah, jo šele takrat lahko začneš dajati drugemu. V zahvalo je dobiš veliko nazaj, a še več ljubezni ne potrebuješ.  Prva lekcija prijateljstva, družine, ljubezni, življenja je, da se naučiš, kako živeti sam. 

Poskusi izkusiti čudež samote. Poskusi biti kdaj sam, v tišini brez aktivnosti. To je bistvo meditacije, biti sam s seboj, brez aktivnosti ... Če postaneš osamljen in začneš pogrešati drugega, potem nekaj manjka tvoji biti; potem še nisi sposoben razumeti, kdo si. 

Bodi potrpežljiv in pojdi globlje v osamljenost. Razvij močno, štiriindvajset ur prisotno, aktivno čuječnost, takrat ko sanjaš in takrat ko si buden. Osamljenost se bo s konstantnim vsakodnevnim trudom s časom spremenila v samoto. Transformira se, ker je nasprotni, obratni vidik enakega pojava. Osamljenost je nasprotni vidik samote. Če boš šel globlje v osamljenost, bo trenutek zagotovo prišel, ko boš začel čutiti njene pozitivne vidike. Nasprotni pojavi so zmeraj tesno povezani med sabo. Dan ne obstaja brez noči, vrh ne obstaja brez doline, kakor tudi osamljenost ne obstaja brez samote. Če izkušaš močno žalost, depresijo, obup, potem si lahko prepričan, da obstaja nasprotno stanje, v katerem izkušaš močno srečo, ekstazo, optimizem. 

Torej bodi osamljen, trpi osamljenost. Težko ti bo, meditacija je težka. Ljudje pridejo in me prosijo: \textquote{Da, pripravljeni smo sedeti, toda daj nam mantro, da lahko mantramo.} Veš, kaj me v resnici prosijo? Prosijo me, da ne bi bili sami, nočejo se soočiti s svojo osamljenostjo. Mantrali bodo mantro, ona bo postala njihova družba. Peli bodo: \textquote{Ohm, ohm, ohm.} To jih bo potolažilo. Zvok \textquote{ohm} bo postal njihov prijatelj. 

Bistvo meditacije so izgubili. Ker meditacija preprosto pomeni biti sam s seboj, brez kakršnekoli aktivnosti. Mantranje je trik uma, s katerim usmeriti pozornost stran od svoje osamljenosti. To um zmeraj dela. Si opazil, kaj um počne, medtem ko si sam? Fantazira o prihodnosti, melje stare spomine, načrtuje, kaj bo naredil v naslednjem trenutku. Zmeraj nekaj počne, da ti preusmeri pozornost tako, da nisi nikoli polno prisoten v tem trenutku. 

Kadarkoli si sam, se tvoj um zateče v fantazije in stare spomine ter odpotuje daleč, daleč stran od sedanjosti. Ko nimaš česa početi in ti je dolgočas, se prepustiš, da te um ukrade in odpelje na fantastično potovanje. 

To se zgodi z osebo, ki gre v puščavo, dobila bo privide, saj je puščava zelo monotona. Tam ni ničesar, na kar bi se lahko um fokusiral, nobenih distrakcij, neskončna monotonost peska. Človek postane zasanjan, začne nadomeščati. Če um ne najde ničesar zanimivega v resničnem svetu, ustvari svoj notranji fantazijski svet, v katerega začne vstopati. 

Enaka usoda doleti ljudi, ki odidejo meditirat na osamljen kraj. Po nekem času dobijo privide. Dobijo videnja o duhovih in angelih, bogovih in boginjah. Vidijo vse tisto, kar so se naučili in se jim je najmočneje vtisnilo v spomin. Če veruješ v Boga, se bo po nekem času meditacije pojavil pred teboj in to bo čisti privid. Ko ni ničesar, kar bi zamotilo um, um sam ustvari svoj sanjski svet. Če se prividi ponavljajo, se počasi zlijejo skupaj s tvojo resničnostjo. 

Veliko poskusov je bilo narejenih na ljudeh, ki so jim zmanjšali čutno zaznavanje. Človeku so med poskusom pokrili oči, zatesnili ušesa, celotno telo zavili v peno. Dotik pene je monoton, tema v očeh je monotona, brezzvočje je monotono, v dveh, treh urah je človek vstopil v svoj fantastični sanjski svet ... Če so poskus nadaljevali, je oseba po treh tednih znorela. Njegova domišljija je prevzela nadzor nad razumom. 

Ampak zakaj začne um fantazirati? Ker ni sposoben živeti sam s sabo v samoti. Ali potrebuje distrakcijo v resničnosti, če v resničnosti ni ničesar, potem ustvari svojo fantazijsko resničnost. Fantazija je substitucija dolgočasne resničnosti. Um ne zna živeti v samoti. 

To je razlog, zakaj ponoči sanjaš. Tvoj svet izgine in ostaneš sam. Brskaš po spominih, ustvariš fantazije, začneš sanjati in postaneš nezmožen izkusiti pravi jogijski počitek. Ponoči um sanja, ker nisi sposoben preživeti noči v samoti. Um zamenja monotonost teme s sanjskim svetom, da ni sam. Zakaj so sanje sploh potrebne? Ker bi sicer znorel. 

Ta neverjetna iluzija sanj obstaja samo zato, ker se nisi naučil temeljne stvari - biti sam. Mojster govori resnico. Pravi: \textquote{To je največji čudež. Jaz sedim tu sam s seboj.} Obstajati sam, v ekstazi, brez nepotrebnih fantazij  ... Takrat si prišel domov. Zapolnil si svojo luknjo osamljenosti s samim seboj ... 

Samoto na začetku vidiš kot praznino, a ko vstopiš vanjo, začutiš pravo polnost življenja. 

Ni praznina. Ni praznina, ker vse življenje živiš z drugimi in jih pogrešaš, zato si samoto razlagaš kot praznino. 

Drugih ni, samo ti si, a ne vidiš samega sebe, ker pogrešaš druge. 

Potreba po drugem se ti je zakoreninila v glavo, postala je mehanična navada, v kateri se pojavi občutek, da si prazen, osamljen, ko zraven tebe ni nikogar. Toda če si dovoliš sproščeno pasti v to praznino, slej ko prej spoznaš, da praznina ne obstaja, in takrat odstraniš skupaj s praznino tudi navezanost do drugih stvari, ljudi. Potem se zgodi največji čudež, obstajaš srečen brez razloga. 

Ko je tvoja sreča odvisna od česar-koli, je tudi tvoja nesreča odvisna od te iste stvari. Če si srečen, ker te partner ljubi, boš postal nesrečen, ko te ne bo ljubil več. Če si srečen zaradi katerega-koli razloga, bo ta razlog postal pogoj tvoje sreče. Tvoja sreča bo zmeraj odvisna od zunanjih dejavnikov, živel boš kot ladjica sredi nevihte. Postal boš berač svoje sreče. Svet se spreminja, v vsakem trenutku se lahko vse spremeni. Tvoja sreča mora biti odvisna samo od tebe, sicer ne boš nikoli vedel, ali si srečen ali nesrečen. 

Nikoli ne moreš biti prepričan o ničemer. Tvoj partner je zdaj prijeten, a čez čas bo mogoče postal zagrenjen in jezen. Včeraj ti je govoril o ljubezni, a danes je izgubil potrpljenje. 

Zanašati se na drugega, je suženjstvo, tvoja sreča bo zmeraj odvisna od drugih, nikoli ne boš mogel čutiti ekstaze. 

Prava sreča je mogoča samo v popolni, brezpogojni svobodi. Jogiji ji pravijo moksha. Moksha pomeni absolutna svoboda. Biti sam s seboj v samoti je moksha, ker nisi več odvisen od česarkoli. Tvoja sreča je samo tvoja, ne izposojaj in ne zahtevaj je od nikogar. Nihče ti je ne more vzeti, niti smrt. 

Smrt te odvzame od drugih, nikoli pa te ne odvzame od samega sebe. Smrt se sliši strašljiva, ker te oddalji od ljudi, ki jih imaš rad in oni tebe - žena ne bo več z možem, mati ne bo več z otroki. Smrt te odvzame samo od drugih. A ne more te odvzeti od samega sebe, to ni mogoče. 

Ko se naučiš biti sam s seboj v samoti, je smrt nesmiselna, strah pred smrtjo zate ne obstaja več. Postaneš nesmrten. Smrt ti ne more odvzeti ničesar več. To, kar ti smrt odvzame, predaš že veliko prej prostovoljno. 

Točno to v meditaciji ljudje nezavedno naredijo, predajo neesencijalne stvari, ki so pogoj za njihovo srečo. Sreča tako ni več odvisna od stvari, ki ti jih lahko življenje kadarkoli odvzame. To, kar smrt naredi, pravi jogi v meditaciji naredi sam od sebe, prostovoljno. Dobro ve, da mu bo slej ko prej vse odvzeto, to jogi preda sam od sebe, še preden se zgodi. Lahko živi na cesti kot berač, a v resnici živi kot kralj svojega notranjega sveta. 

Zelo lepo je biti sam. Nič se ne more primerjati s tem. Njena lepota je ultimativna, njena veličina neskončna, njena moč bogovska. 

Čas je, da prideš domov. Najdi pot, začni hoditi, a vedi, da boš na njej trpel osamljenost. Trpel jo boš, a pojdi skozi. Za ekstazo, ki bo nekoč tvoja, boš moral trpeti. A na koncu te čaka ultimativna svoboda. Tvoje kraljestvo, tvoj dom, v katerem trenutno manjkaš ti. 