        

   Zmeraj cutis potrebo po druzbi. Potreba je ogromna, nekaj ti manjka. V sebi imas luknje, polnes jih z drugo prisotnostjo, ta te nekako izpopolni, drugace pa si nepopoln. 

Brez druzbe neves kdo si, izgubis svojo identiteto. Ona je tvoje ogledalo v katerem se lahko gledas. Brez druzbe si kar naenkrat prepuscen samemu sebi, takrat zacutis nelagodje ker ne ves kdo si. Ko si sam si v zelo neprijetni, mucni druzbi. Ne ves s kom si, ker ne ves kdo si. 

Z drugo stvarjo in osebo si si zgradil svoj jaz, svojo osebnost. Ves komu pripadas, sluzis, kaj ljubis, sovrazis, verujes. Z njo si si definiral svoj jaz zato ves tocno kdo si! Kdo si, ko zraven tebe ni nicesar? 

Globoko v sebi cutis praznino... nedolocljivost. Tam je luknja... nicnost. Njena gravitacija te vlece v njo. Zacutis strah. Tvoj jaz se izgublja v nicnosti, postanes panicen. Hitis v druzbo. Ona te resi pekla. Spet cutis svoj jaz. Zdaj ga spet lahko vidis! Resil si ga. Ko si sam si spet prepuscen svoji praznini, nicnosti. 

Nihce si ne zeli biti sam. Tvoj najvecji strah je da bi nekoc ostal sam. 

Naredis tisoc in eno stvar, samo da ne bi ostal sam. Posnemas druge, da si jim vsec, da se druzijo s tabo, da nisi vec osamljen. Izgubis svojo individualnost, unikatnost. Postanes imitator, ker drugace bi ostal sam. 

Postanes del druzine, organizacije, religije, ideologije. Zelis postati del necesa, da se lahko sprostis, da lahko odmislis svojo osamljenost. Toliko \textquote{podobnih}, enako mislecih ljudi. Zraven njih, nisi vec osamljen. 

Biti sam je res najvecji cudez. To pomeni da ne pripadas nobenemu cloveku, nobeni druzini, nobeni organizaciji, nobeni religiji, nobeni ideologiji, nobenemu cilju. Ne pripadas nikomur, ne pripadas nicemur. Ne pripadas... samo obstajas, samo si. Takrat si se naucil kako ljubiti svojo nedefinirano, nedolocljivo osebnost. Spoznal si kako biti sam s seboj. 

Potrebe po druzbi so izginile. Nimas vec miselnih vojn, nimas vec eksistencnih kriz identitete, nimas vec straha pred smrtjo, nicesar vec ne pogresas. Nimas vec pomankjivosti... srecen si sam s seboj. Ne potrebujes nicesar, zdaj je tvoja sreca brezpogojna. Da, to je najvecji cudez. 

Toda, mojster pravi, \textquote{Sedim sam s seboj.} Ko si sam, nisi sam, si samo osamljen. Razlika med osamljenostjo in samoto je ogromna. Ko si osamljen mislis na drugo prisotnost. Takrat pogresas druzbo. Osamljenost je negativno, ne-cujecnostno umsko stanje. 

Cutis, razmisljas da bi bilo bolje, ce bi bilo zraven tebe se nekaj - prijatelj, druzina, ljubljena oseba, glasba, narava, vznemirjenje, drama, akcija... karkoli... Bilo bi lepo ce bi bilo, zraven tebe se nekaj, a tega ni. 

V osamljenosti pogresas nekaj kar trenutno ni zraven tebe, ko si osamljen si jokajoci otrok ki si zeli nekaj kar trenutno nima v svojih rokah. Samota, pa je prisotnost samega sebe. 

Samota je pozitivno, zelo cujecnostno umsko stanje. Samota je opazovanje samega sebe, izskustvo sebe v tem trenutku. Druzenje z samim seboj.  Z samim seboj lahko napolnis svoje lastne luknje. Z samim seboj lahko napolnis prostor kjer bi moral biti drugi. Cela vesolja lahko nasicis s samim seboj in takrat ne pogresas nicesar vec. 

Ce bi ti ves svet izginil, ne bi v tej praznini pogresal nicesar.  Ljubil bi to velikansko praznino, cisto neskoncnost v kateri bi bil ti. Nicesar ne bi pogresal ker bi takrat prisel domov. Vedel bi da si ti, sam sebi dovolj, da je to vse kar v resnici rabis. 

To pa se ne pomeni da moz, ki zna ziveti sam, ne zna ziveti z drugimi. V bistvu je samo on sposoben ziveti z drugimi. Ker zna ziveti sam s sabo, zna ziveti z drugimi. Ce nisi sposoben biti sam s seboj, kako si lahko sposoben ziveti z drugimi? Samo TI poznas sebe najbolje. Ce samim sabo ne znas ziveti, v globoki ljubezni, exstazi, kako lahko pricakujes od drugih, ki te ne poznajo tako kot ti poznas sebe, da ti bodo dajali to kar si TI se sam sebi ne znas dati? Drugi ne ve! Drugi ne ve. Drugi ne ve... 

Clovek ki ljubi svojo samoto je sposoben ljubezni, clovek ki cuti osamljenost pa je ni. Clovek ki je srecen sam z seboj je poln ljubezni. Ne potrebuje ljubezni od kogarkoli, a se vedno jo daje. Ko si v pomanjkanju necesa kako lahko dajas to drugemu? Ko si v pomanjkanju si berac. Ko imas ljubezen v neomejenih kolicinah jo lahko das tudi drugemu. V zahvalo je dobis veliko nazaj, a se vec ljubezni ne potrebujes.  Prva lekcija prijateljstva, druzine, ljubezni, zivljenja, je da se naucis kako ziveti sam. 

Probaj izskusiti cudez samote. Probaj biti sam, v tisini brez aktivnosti, da zacnes pocasi spoznavati samoto ki osvobaja. To je bistvo meditacije, biti sam z seboj, brez aktivnosti. Probaj jo izskusiti... Ce postanes osamljen da zacnes pogresati drugega, potem nekaj manjka tvoji biti, potem se nisi sposoben razumeti kdo si. 

Bodi potrpezljiv in pojdi globje v osamljenost.. Razvij mocno, 24h prisotno, aktivno cujecnost, takrat ko sanjas in takrat ko si buden. Osamljenost se bo z konstantnim vsakodnevnim trudom s casom spremenila v samoto. Transformira se ker je nasprotni, obratni vidik enakega pojava. Osamljenost je nasprotni vidik samote. Ce bos sel globje v osamljenost bo moment zagotovo prisel ko bos zacel cutiti njene pozitivne vidike. Nasprotni pojavi so zmeraj tesno povezani med sabo. Dan ne obstaja brez noci, vrh ne obstaja brez doline, kakor tudi osamljenost ne obstaja brez samote. Ce izskusas mocno zalost, depresijo, obup, potem si lahko preprican da obstaja nasprotno stanje v katerem izskusas mocno sreco, ekstazo, optimizem. 

Torej bodi osamljen, trpi osamljenost. Tezsko ti bo, meditacija je tezska. Ljudje pridejo in me prosijo, \textquote{Da, pripravljeni smo sedeti, toda daj nam mantro da lahko mantramo.} Ves kaj me v resnici prosijo? Prosijo me da ne bi bili sami, nocejo se soociti z svojo osamljenostjo. Mantrali bodo mantro, ona bo postala njihova druzba. Peli bodo, \textquote{Ohm, Ohm, Ohm}, to jih bo potolazilo ker zdaj niso vec sami. Zvok \textquote{Ohm} bo postal njihov prijatelj. 

Bistvo meditacije so izgubili. Ker meditacija preprosto pomeni biti sam s seboj, brez kakrsne koli aktivnosti. Mantranje je trik uma s katerim ti usmeriti fokus stran od tvoje osamljenosti. To um zmeraj dela. Si opazil kaj um pocne medtem ko si sam? Fantazira o prihodnosti, melje stare spomine, planira kaj bo naredil v naslednjem trenutku. Zmeraj nekaj pocne da ti preusmeri pozornost tako da nisi nikoli polno prisoten v tem trenutku. 

Kadarkoli si sam, se tvoj um zatece v fantazije in stare spomine ter odpotuje dalec, dalec stran od sedanjosti. Ko nimas nicesar za delati in ti je dolgocas, se prepustis da te um ukrade in odpelje na fantasticno potovanje. 

To se tudi zgodi z osebo ki gre v puscavo, dobila bo privide, saj je puscava zelo monotona. Tam ni nicesar, na kar bi se lahko um fokusiral, nobenih distrakcij, neskoncna monotonost peska. Clovek postane zasanjan, zacne nadomescati. Ce um ne najde nicesar zanimivega v realnem svetu, ustvari svoj notranji fantazijski svet v katerega zacne vstopati. 

Enaka usoda doleti ljudi ki odidejo meditirat na osamljen kraj. Po nekem casu dobijo privide. Dobijo videnja o duhovih in angelih, bogovih in boginjah. Vidijo vse tisto kar, so se naucili in se jim je najmocneje vtisnilo v spomin. Ce verujes v Boga se bo po nekem casu meditacije pojavil pred teboj, in to bo cisti privid. Ko ni nicesar kar bi zamotilo um, um sam ustvari svoj sanjski svet. Ce se prividi ponavljajo, se pocasi zlijejo skupaj z tvojo realnostjo. 

Veliko poskusov je bilo narejenih na ljudeh, katerim so zmanjsali cutno zaznavanje. Cloveku so med poskusom prikrili oci, usesa zatesnili, celotno telo zavili v peno. Dotik pene je monoton, tema v oceh je monotona, brezzvocje je monotono, v dveh, treh urah je clovek vstopil v svoj fantasticni sanjski svet... Ce so poskus nadaljevali je oseba po treh tednih znorela. Njegova domislija je prevzela kontrolo nad njegovim razumom. 

Ampak zakaj um zacne fantazirati? Ker ni sposoben ziveti sam z sabo v samoti. Ali potrebuje distrakcijo v realnosti, ce v realnosti ni nicesar, potem ustvari svojo fantazijsko realnost. Fantazija je substitucija, dolgocasne realnosti. Um ne zna ziveti v samoti. 

To je razlog zakaj ponoci sanjas. Tvoj svet izgine in ostanes sam. Brskas po spominih, ustvaris fantazije, zacnes sanjati, in postanes nezmozen izskusiti pravega jogijskega pocitka. Ponoci um sanja ker nisi sposoben preziveti noci v samoti. Um zamenja monotonost teme z sanjskim svetom da ni sam. Zakaj so sanje sploh potrebne? Ker bi drugace znorel. 

Ta neverjetna iluzija sanj obstaja samo zaradi tega ker se nisi naucil te temelne stvari - biti sam. Mojster govori resnico. Pravi, \textquote{To je najvecji cudez. Jaz sedim tu sam s seboj.} Obstajati sam, v ekstazi, brez nepotrebnih fantazij... Takrat si prisel domov. Zapolnil si svojo luknjo osamljenosti z samim seboj... 

Samoto na zacetku vidis kot praznino, a ko vstopis vanjo zacutis pravo polnost zivljenja. 

Ni praznina. Ni praznina ker vse zivljenje zivis z drugimi in jih pogresas, zato samoto interpretiras kot praznino. 

Drugih ni, samo ti si, a ne vidis samega sebe, ker pogresas druge. 

Potreba po drugem se ti je zakoreninila v glavo, postala je mehanicna navada v kateri se pojavi obcutek da si prazen, osamljen, ko zraven tebe ni nobenega. Toda ce si dovolis sprosceno pasti v to praznino, slej kot prej spoznas da praznina ne obstaja, in takrat odstranis skupaj z praznino tudi navezanost do drugih stvari, ljudi. Potem se zgodi najvecji cudez, obstajas srecen brez razloga. 

Ko je tvoja sreca odvisna od cesar koli, je tudi tvoja nesreca odvisna od te iste stvari. Ce si srecen ker te partner ljubi, bos postal nesrecen ko te nebo ljubil vec. Ce si srecen zaradi katerega koli razloga, bo ta razlog postal pogoj tvoje srece. Tvoja sreca bo zmeraj odvisna od zunanjih dejavnikov, zivel bos kot ladica sredi nevihte. Postal bos berac svoje srece. Svet se spreminja, v vsakem trenutku se lahko vse spremeni. Tvoja sreca mora biti odvisna samo od tebe saj drugace ne bos nikoli vedel ali si srecen ali nesrecen. 

Nikoli ne moras biti siguren o nicemer. Tvoj partner je zdaj prijeten, a cez cas bo mogoce postal zagrenjen in jezen. Vceraj ti je govoril o ljubezni a danes je izgubil potrpljenje. 

Zanasati na drugega je suzenstvo, tvoja sreca bo zmeraj odvisna od drugih, nikoli ne bos mogel cutiti ekstaze. 

Prava sreca je mogoca samo v popolni, brezpogojni svobodi. Jogiji ji pravijo moksha. Moksha pomeni absolutna svoboda. Biti sam s seboj v samoti je moksha ker zdaj nisi vec odvisen od cesarkoli. Tvoja sreca je samo tvoja, ne sposojaj in ne zahtevaj je od nikogar. Nihce ti je ne mora vzeti, niti smrt. 

Smrt te odvzame od drugih, nikoli pa te ne odvzame od samega sebe. Smrt se slisi strasljiva ker te oddalji od ljudi ki jih imas rad in oni tebe - zena ne bo vec z mozem, mati ne bo vec z otroki. Smrt te odvzame samo od drugih. Ampak ne mora te pa odvzeti od samega sebe, to ni mogoce. 

Ko se naucis biti sam s seboj v samoti je smrt nesmiselna, strah pred smrtjo, zate ne obstaja vec. Postanes nesmrten. Smrt ti ne mora odvzeti nicesar vec. To kar ti smrt odvzame, to predas ze veliko prej prostovoljno. 

Tocno to v meditaciji ljudje nezavedno naredijo, predajo neesencijalne stvari katere so pogoj za njihovo sreco. Sreca tako ni vec odvisna od stvari katere ti lahko zivljenje kadarkoli odvzame. To kar smrt naredi, pravi jogi v meditaciji naredi sam od sebe, prostovoljno. Dobro ve da slej ko prej mu bo vse odvzeto, to jogij preda sam od sebe, se preden se to zgodi. Lahko zivi na cesti kot berac a v resnici zivi kot kralj svojega notranjega sveta. 

Zelo lepo je biti sam. Nicesar se ne mora primerjati. Njena lepota ultimativna, njena velicina neskoncna, njena moc bogovska. 

Cas je da prides domov. Najdi pot, zacni hoditi, a vedi da bos na njej trpel osamljenost. Trpel jo bos, a pojdi skozi. Za ekstazo ki bo nekoc tvoja bos moral trpeti. A na koncu te caka ultimativna svoboda. Tvoje kraljestvo, tvoj dom v katerem manjkas ti. 