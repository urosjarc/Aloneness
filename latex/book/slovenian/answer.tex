        

   Zmeraj cutis potrebo po druzbi drugega. Potreba je ogromna, nekaj ti manjka. V sebi imas luknje, polnes jih z prisotnostjo drugega. Drugi te nekako zapolni, drugace si prazen. 

Brez njih neves kdo si, izgubis svojo identiteto. Drugi postanejo ogledalo v katerem se lahko gledas. Brez njih si prepuscen samemu sebi. Zacutis nelagodje, ker ne ves kdo si. Ko si sam si v zelo neprijetni, mucni druzbi. Ne ves s kom si, ker ne ves kdo si. 

Z drugim definiras svoj jaz, svojo osebnost. Ves komu pripadas, sluzis, kaj ljubis, sovrazis, verujes. Z drugo stvarjo in osebo definiras svoj jaz zato ves tocno kdo si! Kdo si, ko zraven tebe ni nicesar? 

Globoko v sebi cutis praznino... nedolocljivost. Tam je luknja... nicnost. Njena gravitacija te vlece v njo. Zacutis strah. Tvoj jaz se izgublja v nicnosti, postanes panicen. Hitis v druzbo. Drugi te resi pekla. Tvoj jaz je spet definiran. Resil si ga. Ko si sam si spet prepuscen svoji praznini, nicnosti. 

Ne zelis biti sam. Tvoj najvecji strah je da bi nekoc ostal sam. 

Vse naredis samo da ne bi ostal v sam. Posnemas druge da si jim vsec da nisi vec osamljen. Izgubis svojo individualnost, unikatnost, postanes imitator, ker drugace ostanes sam. 

Postanes pripadnik druzine, organizacije, religije, ideologije. Zelis postati del necesa kjer se lahko sprostis in odmislis svojo samoto, toliko "podobnih" enako mislecih ljudi... nisi vec sam. 

Ostati sam je resnicno najvecji cudez. To pomeni da ne pripadas nobeni cerkvi, ne pripadas nobeni organizaciji, ne pripadas nobeni teologiji, ne pripadas nobeni ideologiji, ne pripadas nobenemu cloveku - ne pripadas nicemur, preposto obstajas brez razloga, samo si. Takrat si se naucil, kako ljubiti svojo nedefinirano, nedolocljivo realnost. Spoznal si kako biti sam s seboj. 

Potrebe po druzbi drugega so izginile. Nimas vec miselnih vozlov, nimas vec eksistencnih lukenj, nicesar vec ne pogresas, nimas vec pomankjivosti - srecen si sam sabo. Ne potrebujes vec nicesar, tvoja sreca nima vec pogojev. Da, to je resnicno najvecji cudez. 

Toda ne pozabi, mojster pravi, \textquote{Sedim sam s seboj.} Ko si sam nisi v samoti, si samo osamljen - razlika med osamljenostjo in samoto je velika. Ko si osamljen mislis na drugega, pogresas drugega. Osamljenost je negativno, nezazeleno stanje. 

V osamljenosti cutis da bi bilo bolje ce bi bil zraven tebe se nekdo - prijatelj, ljubljena oseba. Bilo bi lepo ce bi bil zraven se nekdo, a ni nobenega. 

Ko si osamljen pogresas drugo osebo. Samota pomeni prisotnost samega sebe. 

Samota je zelo pozitivno, zazeleno stanje. Je tvoja prisotnost ki prekipeva. Tako si je poln da bi lahko z njo napolnil celotno vesolje, in tam ni potrebe po nicemer. 

Ce ves svet izgine, zen mojster ne bi pogresal nicesar. Ce se kar naenkrat zaradi carovnije vse izbrise iz vesolja in mojster ostane sam, bo srecen kot prej, nicesar ne bo pogresal. Ljubil bo to neskoncno praznino, cisto neskoncnost. Nicesar ne bi pogresal ker je prisel domov. Ve da je on, sam sebi dovolj, to je vse kar v resnici rabi. 

To pa ne pomeni da on ki je postal razsvetljen in ki je prisel domov ne zna ziveti z drugimi. V resnici je samo on sposoben biti z drugimi. Ker zna ziveti sam, zna ziveti z drugimi. Ce ne znas ziveti sam s seboj, kako lahko zivis z drugimi? Samo ti ves kaj v resnici TI potrebujes. Ce sam sabo ne znas biti, v globoki ljubezni, radosti - kako si lahko z drugim? Drugi lahko samo ugibajo kaj TI v resnici potrebujes. 

Clovek ki ljubi svojo samoto je sposoben ljubezni, clovek ki cuti osamljenost pa je ni. Clovek ki je srecen sam z seboj je poln ljubezni prekipeva in se izliva iz njega. Ne potrebuje ljubezni od nikogar, a se vedno jo daje. Ko si v pomanjkanju kako lahko kar koli das komurkoli? Ko si v pomanjkanju si berac. Ko lahko dajes, dobis veliko nazaj. To je odziv, naravni odziv. Prva lekcija ljubezni je da se naucis ziveti sam s seboj. 

Poskusi obcutiti. Vsedi se kdaj sam s seboj. To je bistvo meditacije - biti sam z seboj, delati nicesar, probaj poskusiti. Ce zacutis osamljenost potem nekaj manjka tvoji biti, potem se nisi sposoben razumeti kdo si. 

Potem pojdi globje v osamljenost dokler ne prides do tocke kjer se osamljenost transformira v samoto. Se transformira - je nasprotni, obratni vidik enakega pojava. Osamljenost je nasprotni vidik samote. Ce bos sel globje v osamljenost bo moment zagotovo prisel ko bos zacutiti njen pozitiven vidik. Ker so se tako nasprotne reci zmeraj tesno povezane. Dan ne obstaja brez noci kakor tudi osamljenost ne obstaja brez samote. 

Bodi osamljen, trpi osamljenost. Tezsko ti bo, meditacija je tezska. Ljudje pridejo do mene in prosijo, \textquote{Da, smo pripravljeni sedeti, toda daj nam mantro da lahko mantramo.} Vem kaj me v resnici prosijo. Prosijo me da ne bi bili sami, nocejo se soociti z svojo osamljenostjo. Mantrali bodo mantro - mantra bo postala njihova druzba. Peli bodo, \textquote{Ohm, Ohm, Ohm} - in ne bodo vec sami. Ta zvok \textquote{Ohm} neprestalno ponovljen bo postal njihov prijatelj. 

Bistvo meditacije se jim je izmuznil. Mantranje, sploh ni meditacija, meditacija pomeni biti sam s seboj brez aktivnosti - brez mantranja. To je trik uma. To um zmeraj dela. Ko sedis sam s seboj, si opazil koliko fantazij zacnes videvati?... neskoncne fantazije, sanje. 

Kadarkoli si sam, um zacne sanjariti. Ko nimas nicesar za delati in ti je dolgocas, se zacenjas zatekati v fantaziranje. 

To se zgodi z osebo ki gre v puscavo, zacela bo sanjariti, imela bo videnja, puscava je zelo monotona. Nicesar ni na kar bi se um fokusiral - monotona prostranost peska; brez distrakcij, nicesar novega ni - monotono, dolgocasje. Clovek postane zasanjan, zacne nadomescati. Ce ni nicesar v zunanjem svetu, un ustvari svoj notranji fantazijski svet v katerega zacne gledati. 

To se zgodi z ljudmi ki grejo v jame meditirat. Dobijo videnja. Z umom ustvarijo - bogove in boginje, duhove, angele, Krishno kako igra na flavto, Jezusa - svet lastne ideologije. Ce si bil vzgojen kot Kristjan, slej ko prej bos v himalajski jami spoznal Jezusa, in to bo cisti privid. Nicesar kar bi zamotilo um, um sam ustvari svoj sanjski svet. Ce se to ponavlja dlje casa, postanejo videvanja zelo resnicne. 

Veliko cloveskih eksperimentov je bilo narejenih na zmanjsevanju cutnih drezljajev. Ce so cloveku zmanjsali cutne drezljaje - njegove oci so zaprli, njegova usesa zatesnili, celotno telo zavili v peno da je dotik monoton, tema v oceh monotona, brezzvocje monotono, vse monotono - v dveh, treh urah je clovek zacel sanjariti - fantasticne sanje, in tako resnicne...  Ce so experiment nadaljevali je oseba znorela. Njegova domislija je prevzela kontro uma. 

Zakaj um zacne sanjariti? Zakljucek znanstvenikov je bil da um ni sposoben ziveti sam z sabo. Ali potrebuje nekoga v realnosti, ali, ce v realnosti ni nicesar, um ustvari fantazije. Fantazija je substitucija. Um ne zna ziveti sam. 

To je razlog zakaj ponoci sanjas - ponoci v sanjah si sam; tvoj svet izgine. Tvoje druzine ni vec s tabo, tvojih prijateljev ni, sam si - in postal si nezmozen samote. Um zamenja realnost z sanjskim svetom; sanje, cikli sanj celo noc. Zakaj so sanje sploh potrebne? Ker nisi zmozen biti sam. 

Ta iluzija, ponoci obstaja ker se nisi naucil te temelne stvari - biti sam. Mojster govori resnico. Pravi, \textquote{To je najvecji cudez. Sedim sam s seboj.} Biti sam z sabo v sreci, ekstazi, in hkrati ne zaceti fantazirati... takrat si prisel domov, zapolnil si svoje luknje z samim seboj. 

Ko vstopas v samoto zgleda kot da vstopas v praznino, a ko si notri zacutis polnost, izpolnjenost, razscvetenje, vrhunec. 

Ni praznina. Samo zgleda kot praznina ker vse zivljenje zivis z drugimi in jih pogresas; zato jo interpretiras kot praznino. 

Drugih ni tam, samo ti si - toda ne vidis samega sebe, ker pogresas druge. 

Postal si ustaljen v rutini; ideja drugega se ti je zakoreninila v lastne navade, postala je mehanicna navada, in ko pogresas drugega imas obcutek da si prazen, osamljen, zacnes padati v praznino. Toda ce si pustis pasti v to praznino, slej kot prej spoznas da praznina ne obstaja, in takrat odstranis skupaj z praznino tudi navezanost do drugih. Nato se zgodi najvecji cudez - srecen si brez razloga. 

Ce je tvoja sreca odvisna od drugih stvari, bo tvoja nesreca tudi odvisna od drugih stvari. Ce si srecen ker te partner ljubi, bos postal nesrecen ce te kar naenkrat ne ljubi vec. Ce si srecen zaradi katerega koli razloga, bo ta razlog postal pogoj tvoje srece. Tvoja sreca bo zmeraj odvisna od zunanjih dejavnikov, zmeraj bos kot ladica sredi nevihte, zmeraj bos berac svoje srece. Nikoli ne bos vedel ali si srecen ali nesrecen, svet se spreminja, vsak trenutek ti lahko tla pod nogami izginejo - v vsakem trenutku se lahko vse spremeni. 

Nikoli ne moras biti siguren o nicemer. Partner je zdaj prijeten, a cez cas bo mogoce postal zagrenjen in jezen. Vceraj ti je govoril o ljubezni danes pa je izgubil potrpljenje. 

Zanasati na drugega je veriga - suzenstvo, zmeraj si odvisen od nekoga, nikoli ne moras resnicno cutiti ekstaze. 

Ekstaza srece je mogoca samo v popolni, brezpogojni svobodi. Yogiji ji pravijo moksha. Moksha pomeni absolutna svoboda. Biti sam s seboj je moksha ker zdaj nisi vec odvisen od cesarkoli. Tvoja sreca je samo tvoja, ne sposojaj in ne zahtevaj je od nikogar. Nihce ti je ne mora vzeti, niti smrt. 

Smrt te odvzame od drugih, nikoli te ne odvzame s od samega sebe. Smrt se slisi strasljiva ker te oddalji od ljudi ki jih imas rad in oni tebe - zena ne bo vec z mozem, mati ne bo vec z otroki. Smrt te odvzame samo od drugih. Ne mora te pa odvzeti od samega sebe; to ni mogoce. 

Ko se naucis biti sam s seboj je smrt nesmiselna, strah pred smrtjo ne obstaja vec. Postanes nesmrten. Smrt ti ne mora odvzeti nicesar vec. To kar ti smrt odvzame, to ti predas ze veliko prej prostovoljno. 

Tocno to v meditaciji naredis - predas neesencijalne stvari katere ti smrt lahko odvzame. To kar smrt naredi, yogi v meditaciji naredi sam od sebe, prostovoljno. Dobro ve da slej ko prej mu bo vse odvzeto - to preda sam od sebe. 

Zelo lepo je biti sam. Nicesar se ne mora primerjati z tem. Njena lepota je ultimativna, njegova velicina je velikanska, njena moc ultimativna. 

Cas je da prides domov. A mogel bos trpeti osamljenost pred tem. Trpi jo, pojdi skozi njo. Placal bos za ekstazo ki bo nekoc tvoja - moral jo bos poplacati. Poplacal jo bos z osamljenostjo. Veliko bos pridobil! Pridi domov. 