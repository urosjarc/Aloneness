        

   Zmeraj cutis potrebo po necem. Potreba je velikanska, nekaj ti manjka. V sebi imas luknje; polnes jih z prisotnostjo drugih stvari. One te zapolnijo, drugace si prazen. 

Brez njih neves kdo si, izgubis svojo identiteto. One so ogledalo v katerem se gledas. Brez njih si prepuscen samemu sebi. Zacutis nelagodje, paniko, saj ne ves kdo si. Ko si sam si v zelo neprijetni, mucni druzbi. Ne ves s kom si ker ne ves kdo si. 

Z drugo stvarjo, osebo definiras svoj jaz, svojo osebnost. Ves kateri stvari, osebi pripadas, sluzis, kaj ljubis, sovrazis, verujes. Z drugo stvarjo, osebo definiras svoj jaz zato ves tocno kdo si! Kdo si, ko zraven tebe ni nicesar vec? 

Globoko v sebi cutis praznino... nedolocljivost. Tam je crna luknja... nicnosti. Gravitacija te vlece vanjo. Zacutis strah. Tvoj jaz se izgublja v nicnosti, postanes panicen. Hitis v druzbo. Druga prisotnost te resi pekla. Tvoj jaz je spet definiran. Resil si ga. Ko si sam si spet prepuscen svoji praznini, nicnosti. 

Ne zelis biti sam. Tvoj najvecji strah je da bi nekoc ostal sam v samoti. 

Vse naredis samo da ne bi ostal v sam. Posnemas druge da si jim vsec da nisi vec sam. Izgubis svojo individualnost, unikatnost, postanes imitator, ker drugace ostanes sam. 

Postanes pripadnik druzine, organizacije, religije, ideologije. Zelis postati del necesa kjer se lahko sprostis in odmislis svojo samoto, toliko "podobnih" enako mislecih ljudi... nisi vec osamljen. 

Ostati sam je resnicno najvecji cloveski cudez. To pomeni da ne pripadas nobeni cerkvi, ne pripadas nobeni organizaciji, ne pripadas nobeni teologiji, ne pripadas nobeni ideologiji, ne pripadas nobenemu cloveku - ne pripadas nicemur, preposto obstajas brez razloga, samo si. Takrat si se naucil, kako ljubiti svojo nedefinirano, nedolocljivo realnost. Spoznal si kako biti sam sebi najboljsi prijatelj. 

Potrebe po druzbi drugega so izginile. Nimas vec nobenih miselnih vozlov, nimas vec nobenih eksistencnih lukenj, ne pogresas vec nicesar, nimas vec manjkajocih delov - srecen si da lahko preprosto obstajas. Ne potrebujes nicesar vec, tvoja sreca je brez pogojev. Da, to je resnicno najvecji cloveski cudez. 

Toda ne pozabi, mojster pravi, \textquote{Jaz sedim sam s seboj.} Ko si sam s seboj nisi v samoti, si samo osamljen - in tu je ogromna razlika med osamljenostjo in samoto. Ko si osamljen mislis na drugega, pogresas prisotnost drugega. Osamljenost je negativno stanje. 

V osamljenosti cutis da bi bilo bolje ce bi zraven tebe bil se nekdo drug - tvoj prijatelj, tvoja druzina, ljubljena oseba. Ko si osamljen bi bilo lepo ce bi bil zraven se nekdo drug, vendar ni nobenega. 

Osamljenost pomeni pomanjkanje drugega. Samota pomeni prisotnost samega sebe. 

Samota je zelo pozitivna. Samota je prisotnost, zapolnjena, kipeca, prisotnost. Tako si poln prisotnosti da lahko z njo napolnis celotno vesolje, in tam ni potrebe po nobeni druzbi. 

Ce vse izgine, zen mojster ne bo pogresal nicesar. Ce se kar naenkrat zaradi neke carovnije vse izbrise iz vesolja da ostane samo nicelnost in zen mojster, se zaradi tega dogodka njegova sreca ne bo nic spremenila, nicesar ne bo pogresal. Ljubil bo to neskoncno nicelnost, to cisto neskoncnost. Nicesar ne bo pogresal ker je prisel domov. Ve da je on, sam sebi popolnoma dovolj, to je vse kar v resnici rabi. 

To ne pomeni da moz ki je postal razsvetljen in ki je prisel domov ne zna ziveti z drugimi. V resnici je samo on sposoben ziveti v druzbi drugih. Ker je sposoben biti sam s seboj je sposoben biti z drugimi. Ce nisi sposoben biti sam z seboj, kako si lahko sposoben biti z drugimi? Samo ti ves kaj v resnici TI potrebujes. Ce nisi sposoben biti, v globoki ljubezni sam z saboj, v radosti - kako si lahko z drugimi? Drugi lahko samo ugibajo kaj so tvoje resnice potrebe. 

Clovek ki ljubi svojo samoto je sposoben ljubezni, clovek ki cuti osamljenost ni sposoben ljubezni. Clovek ki je srecen sam z seboj je poln ljubezni, radosti. Ne potrebuje ljubezni od nikogar, a se vedno jo lahko daje drugemu. Ko si v pomanjkanju kako lahko dajes kar koli drugim? Ko si v pomanjkanju si berac. Ko lahko dajes drugim, veliko ljubezni dobis nazaj. To je naravni odziv drugega. Prva lekcija ljubezni je da se naucis ziveti v samoti. 

Probaj jo kdaj obcutiti. Probaj sedeti sam z seboj obcasno. To je bistvo meditacije - samo sedeti sam z seboj, delati nicesar, probaj jo kdaj izskusiti. Ce bom zacutil osamljenost potem nekaj manjka tvojemu obstoju, potem se nisi sposoben razumeti kdo si v resnici. 

Pojdi globje v osamljenost dokler ne prides do tocke kjer se osamljenost transformira v samoto. Transformira se - ker je nasprotni, obratni vidik enakega pojava. Osamljenost je nasprotni vidik samote. Ce bos sel globje v osamljenost bo moment zagotovo prisel ko bos zacel cutiti pozitivene vidike osamljenosti. Ker so se tako nasprotne reci zmeraj tesno povezane skupaj. Dan ne obstaja brez noci kakor tudi osamljenost ne obstaja brez samote. 

Torej bodi osamljen, trpi osamljenost. Seveda ti bo tezsko, meditacija je tezska. Ljudje pridejo do mene in vprasajo, \textquote{Da, smo pripravljeni sedeti, toda daj nam mantro da lahko mantramo.} Kaj me vprasajo? Pravijo da nocejo biti sami s seboj, nocejo se soociti z svojo osamljenostjo. Oni bodo mantrali mantro - mantra bo postala njihova druzba. Mantrali bodo, \textquote{Ohm, Ohm, Ohm} - in ne bodo vec osamljeni. Ta zvok \textquote{Ohm} neprestalno ponovljen bo postal njihov druzabnik. 

Bistvo meditacije so zaradi tega izgubili. Meditacija z mantranjem, sploh ni meditacija, ker meditacija enostavno pomeni biti sam brez aktivnosti - niti mantrati mantro. To je trik uma. To um zmeraj dela. Ko sedis v samoti, ali si kdaj videl koliko fantazij zacnes videvati?... neskoncne fantazije, sanje. 

Kadarkoli si sam, zacnes sanjariti. Ko nimas nicesar za delati in zacutis dolgocasje, takoj se zacenjas zatekati v sanjarjenje. 

Enako se zgodi z osebo ki gre v puscavo, sedeti v saharo, zacel bo sanjariti, imel bo videnja, ker puscava je zelo monotona. Tam ni nicesar na cemer bi se um lahko fokusiral - tam je samo monotona prostranost peska; brez distrakcij, nic novega - monotonicno, dolgocasje. Clovek postane zasanjan, zacne nadomescati. Ce ni nicesar v zunanjem svetu, ustvaris svoj notranji fantazijski svet v katerega zacnes gledati. 

To se zgodi z ljudmi ki grejo v Himalajske jame meditirat. Zacnejo sanjariti. Z umom ustvarijo nato karkoli - bogove in boginje, duhove, angele, Krishno kako igra na flavto, Jezusa - in karkoli kar ustreza domisliji uma, kar koli je tvoja glavna ideologija. Ce si bil ucen v Kristjanstvu, slej kot prej v himalajski jami bos spoznal Jezusa, in to bo cisto sanjarjenje. Ce um nima nicesar kar bi ga lahko zamotilo, um sam ustvari svoj notraji svet sanj. In ko neprestalno sanjaris, ta sanjarjenja zgledajo zelo zelo resnicne. 

Veliko eksperimentov je bilo narejeno na odvzemu cutnega zaznavanja. Ce so cloveku odvzeti vsa cutna zaznavanja - njegove oci zaprte, njegova usesa zatesnjena, celotno telo zavito v peno da je dotik monoton, tema v oceh monotona, brezzvocje monotono, vse monotono - v dveh, treh urah clovek zacne sanjariti - tako fantasticne sanje, in tako resnicne... resnicne od realnosti. In ce cloveku odvzames cutila za 21 dni bo postal nor. Postal bo nor, ker je njegova domislija prevzela kontrolo nad njem. 

Zakaj um zacne sanjariti? Znanstvena razlaga je da um ne mora ziveti sam z sabo. Ali potrebuje nekoga v realnosti, ce v realnosti ni nobenega tam, um ustvari fantazije. Fantazija je substitucija. Um ne mora ziveti sam z sabo. 

To je razlog zakaj sanjas ponoci - ponoci v sanjah si sam; svet izgine. Tvoje druzine ni vec tam, tvojih prijateljev ni, sam si - in postal si nezmozen samote. Tvoj um enostavno zamenja realnost z sanjskim svetom; sanje, cikli sanj celo noc. Zakaj so sanje sploh potrebne? Ker nisi zmozen biti sam s sabo. 

Ta velikanka sanjska iluzija obstaja samo zaradi tega ker se nisi naucil temelne stvari - zivljenja v samoti. Zen mojster je imel prav. On pravi, \textquote{To je najvecji cloveski cudez. Jaz sedim tukaj sam s seboj.} Biti sam s sabo v sreci, ekstazi, in hkrati ne zaceti z fantaziranjem... kar naenkrat si prisel domov, zapolnil si v svojo lastno praznino z svojo prisotnostjo. 

Ko vstopas imas obcutek da vstopas v praznino, vendar ko enkrat vstopis zacutis polnost kaj pomeni BITI, izpolnjenost, razscvetenje, vrhunec. 

Ni praznina. Samo zgleda kot praznina ker vse zivljenje zivimo z drugimi in jih seveda pogresas; zato to interpretiras kot praznino. 

Drugih ni tam, samo ti si - toda ne vidis samega sebe, ker pogresas druge. 

Postal si prevec ustaljen v rutino; ideja drugega se ti je zakoreninila v lastne navade, postala je mehanicna navada, in ko pogresas drugega imas obcutek da si prazen, osamljen, zacnes padati v praznino. Toda ce si pustis pasti v to praznino, slej kot prej spoznas da je praznine ni vec, in z praznino je odsla tudi navezanost do drugih. Nato se zgodi najvecji cloveski cudez - srecen si brez razloga. 

Ne pozabi ce je tvoja sreca odvisna od drugih stvari, bo tvoja nesreca tudi odvisna od drugih stvari. Ce si srecen ker te partner ljubi, bos postal nesrecen ce te kar naenkrat ne bo vec marala. Ce si srecen zaradi katerega koli razloga, ko tega razloga ni vec, bos postal nesrecen. Tvoja sreca bo zmeraj odvisna od zunanjih dejavnikov, zmeraj bos berac svoje srece. Nikoli ne bos siguren ali si srecen ali nesrecen, ker v vsakem momentu se stvari lahko spremenijo, vsak trenutek ti lahko tla pod nogami izginejo - vsak trenutek lahko vse izgine. 

Nikoli ne moras biti popolnoma siguren o nicemer. Partner se ti je prejsnji trenutek smejal, zdaj pa je postal zagrenjen in jezen. Partner ti je vceraj govoril o ljubezni danes pa je izgubil potrpljenje. 

Zanasati na drugega je odvisnost - suzenstvo, zmeraj si odvisen od nekoga, nikoli ne moras resnicno cutiti ekstaze. 

Ekstaza srece je mogoca samo v popolni, brezpogojni svobodi. Na vzshodu temu pravijo moksha. Moksha pomeni absolutna svoboda. Biti pravilno s samim seboj je moksha ker zdaj nisi vec odvisen od kogarkoli in cesarkoli. Tvoja sreca je samo tvoja, ne sposojaj si je od nobenega in nicesar. Nihce ti je ne mora vzeti, niti smrt. 

Zapomni si da te smrt samo odvzame od drugih, nikoli ti ne odvzame samega sebe. Smrt se slisi strasljiva ker te odvzame od ljudi ki jih imas rad in oni tebe - zena ne bo vec z mozem, mati ne bo vec z otroci. Smrt te odvzame samo od drugih. Ne mora te pa odvzeti od samega sebe; to ni mogoce. 

Ko se naucis biti z samim seboj je smrt nesmiselna, nato smrt ne obstaja vec. Postanes brez smrten. Smrt ti ne mora odvzeti nicesar vec od tebe. To kar smrt odvzame, to ji ti predas veliko prej preden se zgodi. 

Tocno to meditacija naredi - v njej predas neesencijalne stvari katere smrt lahko ti odvzame. To kar smrt naredi, yogi v meditaciji naredi sam od sebe, prostovoljno. Dobro ve da slej kot prej mu bo vse odvzeto - to preda sam od sebe. 

Zelo lepo je biti sam z seboj. Nicesar se ne mora primerjati z tem. Njegova lepota je ultimativna lepota, njegova velicina je najvecja velicina, njena moc je ultimativna. 

Cas je da prides domov. In pot je; da bos mogel trpeti osamljenost pred tem. Trpel bos, se vedno vstrajno hodi. Placal bos za ekstazo ki bo nekoc tvoja - moral jo bos poplacati. To trpljenje bo tvoja osamljenost s katero bos placal. Veliko bos pridobil! Pridi domov. 